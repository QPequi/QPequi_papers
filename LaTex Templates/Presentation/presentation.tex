\documentclass[aspectratio=169,t,xcolor=table]{beamer}
\usepackage[utf8]{inputenc}

\usepackage{booktabs} 
\usepackage{subcaption}
\usepackage{blindtext}

\usetheme{Ufg}

% DEFINITIONS
\newtheorem{conj}{Conjecture}
\newtheorem{defi}{Definition}
\newtheorem{teo}{Theorem}
\newtheorem{lema}{Lemma}
\newtheorem{prop}{Proposition}
\newtheorem{cor}{Corollary}
\newtheorem{ex}{Example}
\newtheorem{exer}{Query}

\setbeamertemplate{theorems}[numbered]
\setbeamertemplate{caption}[numbered]

% PRIMARY DEFINITIONS

\setPrimaryColor{Black} % Set the default Color

% First one is logo in title slide and second one is the logo used in the remaining slides
\setLogos{lib/logos/qpequi.png}{lib/logos/qpequi.png} 

\begin{document}

%----------------------------------------
% TITLE SLIDE
%----------------------------------------
\title{Title of the talk}
\subtitle{Subtitle}

\author{Speaker}

\institute[UFG]
{
  Institute of Physics\\
  Federal University of Goiás
}
\date{2021}

\frame[noframenumbering]{\titlepage}


%----------------------------------------
% FIRST SLIDE - TABLE OF CONTENTS
%---------------------------------------- 
\setLayout{vertical} % Define the layout.

\begin{frame}
    \frametitle{Table of Contents}
    \tableofcontents
\end{frame}


%----------------------------------------
% BODY OF THE PRESENTATION
%----------------------------------------

\section{Common presentation elements}

\subsection{Box}

\setLayout{vertical} % Set the layout

%---------------------------------------------------------
\begin{frame}{Example on using box}

    \footnotesize
    
    \begin{ex}
        Some text here
    \end{ex}
    
    \begin{block}{Solution}
        Some text here
    \end{block}

\end{frame}
%---------------------------------------------------------

\subsection{Table}

%---------------------------------------------------------
\begin{frame}{Example on using table}

    \begin{table}[]
        \centering
        \caption{\label{tab:1}Countries and their codes}
        
        \renewcommand{\arraystretch}{1.5}
        \setlength{\tabcolsep}{10pt}
        
        {\rowcolors{2}{}{LightGray!10}
            \begin{tabular}{ p{3cm}p{3cm}p{3cm}  }
                \toprule 
                \textbf{Country Name} & \textbf{Code 2} & \textbf{Code 3} \\
                \midrule
                Afghanistan & AF &AFG \\
                Aland Islands & AX   & ALA \\
                Albania &AL & ALB \\
                Algeria    &DZ & DZA \\
                \bottomrule
            \end{tabular}
        }
    \end{table}
    
\end{frame}
%---------------------------------------------------------

\subsection{Image}

%--------------------------------------------------------- 
\begin{frame}{Example on using image}

    \begin{figure}
        \centering
        \includegraphics[width=.7\textwidth]{figures/mafalda.jpg}
        \caption{The great Mafalda!}
        \label{fig:mafalda}
    \end{figure}
    
\end{frame}
%---------------------------------------------------------

\section{Changing colors and Layouts}

\setLayout{blank}     % Changing layout

\setBGColor{Black}  % Changing background color 

%---------------------------------------------------------
\begin{frame}{Clean layout and two-column text}
    
    \begin{columns}
    
        \column{0.5\textwidth}
        This is a text in first column.
        $$E=mc^2$$
        $$ 1 + 2 + \cdots + k =  \frac{k \cdot (k + 1)}{2}.$$
        \begin{itemize}
        \item First item
       
        \item Second item
        \end{itemize}
        
        \column{0.5\textwidth}
        This text will be in the second column
        and on a second tought this is a nice looking
        layout in some cases.
        
        \begin{enumerate}
            \item First
            \item Second
        \end{enumerate}
        
    \end{columns}
    
\end{frame}
%---------------------------------------------------------

%---------------------------------------------------------
\setLayout{vertical}      % Back to the original layout.
\setBGColor{PrimaryColor} % Back to the original colour.

\begin{frame}{Sample frame title}
    
    In this slide, some important text will be
    \alert{highlighted} because it's important. Please, don't abuse it.
    
    \begin{block}{Remark}
        Sample text
    \end{block}
    
    \begin{alertblock}{Important theorem}
        Sample text in alert box
    \end{alertblock}
    
    \begin{examples}
        Sample text in green box. The title of the block is ``Examples".
    \end{examples}
    
\end{frame}
%---------------------------------------------------------

\section{Main point layout}

\setLayout{mainpoint} % Change layout again.
\setBGColor{Black}  % Change color again.

%---------------------------------------------------------
\begin{frame}{}
    \frametitle{\textcolor{Pequi}{Preliminary Empirical Study}}
\end{frame}
%-------------------------------------------------------

\setLayout{horizontal}       % New layout
\setBGColor{PrimaryColor}    % Back to the original color

%-------------------------------------------------------
\begin{frame}
    \frametitle{Sample frame title}
    This is a text in second frame. For the sake of showing an example.
    
    \begin{itemize}
        \item<1-> Text visible on slide 1
        \item<2-> Text visible on slide 2
        \begin{itemize}
            \item text subitem
        \end{itemize}
        \item<3> Text visible on slides 3
        \item<4-> Text visible on slide 4
    \end{itemize}
\end{frame}
%---------------------------------------------------------


%----------------------------------------
% LAST SLIDE - THANKS
%----------------------------------------
\setLayout{blank}
\begin{frame}
    
    \centering
    \vspace{2cm}
    
    \textbf{\Huge Thank you for your attention}
    
    \ \\
    
    \textbf{Contact}
    \ \\
    
    \text{\footnotesize speaker@qpequi.com}
    
    \vspace{2cm}
    \begin{figure}
        \centering
        \begin{subfigure}{0.2\textwidth}
            \centering
            \hspace{-2cm}\includegraphics[height=1.5cm]{lib/logos/qpequi.png}
        \end{subfigure}%
        \qquad 
        \begin{subfigure}{0.2\textwidth}
            \centering
            \includegraphics[height=1.5cm]{lib/logos/ufgw.png}
        \end{subfigure}
      
    \end{figure}
    
\end{frame}


\end{document}